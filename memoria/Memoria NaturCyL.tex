
\documentclass[a4paper]{article}
\usepackage[spanish]{babel}
\selectlanguage{spanish}
\usepackage[utf8]{inputenc}
\usepackage[T1]{fontenc}



\usepackage[a4paper,top=3cm,bottom=2cm,left=3cm,right=3cm,marginparwidth=1.75cm]{geometry}
\usepackage{amsmath, amsthm, amsfonts}
\usepackage{graphicx}
\usepackage[colorinlistoftodos]{todonotes}
\usepackage[colorlinks=true, allcolors=blue]{hyperref}

\title{Memoria del proyecto NaturCyL}
\author{David Población Criado\\
}

\begin{document}
\maketitle

\begin{abstract}
Aplicación Android para consultar información acerca de los espacios naturales de Castilla y León, así como sus equipamientos.
\end{abstract}

\section{Conjuntos de datos}
Los datos, extraídos del portal de Datos Abiertos de la Junta de Castilla y León, que se pueden consultar en esta aplicación son los siguientes:
\begin{itemize}
	\item Aparcamientos en espacios naturales
        \item Zonas recreativas en espacios naturales
        \item Árboles singulares en espacios naturales
        \item Espacios naturales
        \item Casas del parque en espacios naturales
        \item Refugios en espacios naturales
        \item Zonas de acampada en espacios naturales
        \item Centros de visitantes en espacios naturales
        \item Observatorios en espacios naturales
        \item Miradores en espacios naturales
        \item Quioscos en espacios naturales
        \item Sendas en espacios naturales
\end{itemize}

Para la representación de estos datos se ha seguido el siguiente diagrama de clases UML:

\section{Pantallas}


\end{document}